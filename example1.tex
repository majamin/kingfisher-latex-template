\documentclass{article}
\usepackage{kingfisher}

% Used for nice tables:
\usepackage{float}
\usepackage{booktabs}
\usepackage{siunitx}

\settitle{Example One}
\setauthor{Mr. Soand So}
\setdate{\today}

% Question counter
\newcounter{qscount}

% Question environment
\newenvironment{qs}{
  \refstepcounter{qscount}
  \par\medskip
  \noindent\textbf{Question~\theqscount.}~
}{\par\medskip}

\begin{document}

\section{Limits and the tangent problem}

Here is a sample paragraph introducing the document.

\begin{example}[Completing the Square for $x^2 + 6x + 5$]
	To complete the square:
	\begin{align}
		x^2 + 6x + 5
		 & = x^2 + 6x + 9 - 9 - 4 \label{eq:prfsq} \\
		 & = (x + 3)^2 - 13 \label{eq:factor}
	\end{align}

	In step~(\ref{eq:prfsq}), we halve 6 and square the answer to find the term that will produce a perfect square trinomial.
	We add and subtract this value, 9, and complete the factorization in step~(\ref{eq:factor}).

\end{example}

\subsubsection*{More Details}

A subsection for more math.

\[
	\int_0^\infty e^{-x^2} dx = \frac{\sqrt{\pi}}{2}
\]

A nice table:

% Requires float, booktabs, and siunitx packages
\begin{table}[htbp]
	\centering
	\caption{Comparison of Algorithm Performance on Sample Dataset}
	\label{tab:algo-performance}
	\vspace{5pt}
	\begin{tabular}{@{} l S[table-format=2.1] S[table-format=1.2] @{}}
		\toprule
		Algorithm      & {Accuracy (\%)} & {Runtime (s)} \\
		\midrule
		Decision Tree  & 85.3            & 0.12          \\
		Random Forest  & 92.1            & 0.45          \\
		SVM            & 89.7            & 0.78          \\
		Neural Network & 94.5            & 1.34          \\
		k-NN           & 88.2            & 0.59          \\
		\bottomrule
	\end{tabular}
\end{table}

\begin{qs}
	What is the value of the derivative of $f(x) = x^2$ at $x = 3$?
\end{qs}

\subsection{Simple Python Function}

Here is an example of a Python function:

\begin{codeblock}{python}{Python example}
	def fibonacci(n):
	a, b = 0, 1
	for _ in range(n):
	print(a)
	a, b = b, a + b
\end{codeblock}

\subsection{Question Environment}

\begin{qs}
	What is the value of the derivative of $f(x) = x^2$ at $x = 3$?
\end{qs}

\end{document}

